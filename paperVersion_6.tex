\documentclass{article}

%%%%
% babel
% PLOTS mapas y conglomerados
% bibliografia
%%%%
\usepackage[utf8]{inputenc}
\usepackage{longtable}
\usepackage{authblk}
\usepackage{adjustbox}

\usepackage{natbib} %para mejorar la bibliografía

%\usepackage[spanish]{babel} %para mejorar el español
%\renewcommand\spanishtablename{Tabla}


\title{LOS INDICES DEL MUNDO}
% autores
\renewcommand\Authand{, y }
\author[1]{\normalsize Estrella DelCurso}
\author[2]{\normalsize Prossimo Deal Lado}

\affil[1,2]{\small  Escuela de IngenierÃ<U+00AD>a,Universidad de la vida\\
\texttt{{delcurso,deallado}@vida.edu}}
\affil[1]{\small Instituto de altas investigaciones financieras\\
Banco del Parque\\
\texttt{delcurso@bp.com}}

\date{}

%%%%
\usepackage{Sweave}
\begin{document}
\input{paperVersion_6-concordance}

\maketitle


\begin{abstract}
Este es mi primer trabajo en exploracion y modelamiento de indices usando LATEX. Este trabajo lo he hecho bajo la filosofÃ<U+00AD>a de trabajo replicable. Este es mi primer trabajo en exploracion y modelamiento de indices usando LATEX. Este trabajo lo he hecho bajo la filosofÃ<U+00AD>a de trabajo replicable. Este es mi primer trabajo en exploracion y modelamiento de indices usando LATEX. Este trabajo lo he hecho bajo la filosofÃ<U+00AD>a de trabajo replicable. Este es mi primer trabajo en exploracion y modelamiento de indices usando LATEX. Este trabajo lo he hecho bajo la filosofÃ<U+00AD>a de trabajo replicable.
\end{abstract}

\section*{Introducción}

Aqui les presento mi investigacion sobre diversos indices sociales en el mundo. Los indices los conseguÃ<U+00AD> de wikipedia, espero que les gusten mucho. Aqui les presento mi investigacion sobre diversos indices sociales en el mundo. Los indices los conseguÃ<U+00AD> de wikipedia, espero que les gusten mucho.Aqui les presento mi investigacion sobre diversos indices sociales en el mundo. Los indices los conseguÃ<U+00AD> de wikipedia, espero que les gusten mucho.Aqui les presento mi investigacion sobre diversos indices sociales en el mundo. Los indices los conseguÃ<U+00AD> de wikipedia, espero que les gusten mucho.
Aqui les presento mi investigacion sobre diversos indices sociales en el mundo. Los indices los conseguÃ<U+00AD> de wikipedia, espero que les gusten mucho.Aqui les presento mi investigacion sobre diversos indices sociales en el mundo. Los indices los conseguÃ<U+00AD> de wikipedia, espero que les gusten mucho.Aqui les presento mi investigacion sobre diversos indices sociales en el mundo. Los indices los conseguÃ<U+00AD> de wikipedia, espero que les gusten mucho.

Comencemos viendo que hay en la sección \ref{univariada} en la página \pageref{univariada}.

\clearpage



\section{Exploración Univariada}\label{univariada}

En esta sección exploro cada Ã<U+00AD>ndice. En esta sección exploro cada Ã<U+00AD>ndice. En esta sección exploro cada Ã<U+00AD>ndice. En esta sección exploro cada Ã<U+00AD>ndice. En esta sección exploro cada Ã<U+00AD>ndice. En esta sección exploro cada Ã<U+00AD>ndice. En esta sección exploro cada Ã<U+00AD>ndice. En esta sección exploro cada Ã<U+00AD>ndice. En esta sección exploro cada Ã<U+00AD>ndice.





Para conocer el comportamiento de las variables se ha preparado la Tabla \ref{Tfrecuencias}, donde se describe la distribución de las modalidades de cada variable. Los números representan la situación de algun paÃ<U+00AD>s en ese indicador, donde el mayor valor numérico es la mejor situación.

% latex table generated in R 3.5.1 by xtable 1.8-2 package
% Wed Jul 25 20:36:15 2018
\begingroup\normalsize
\begin{longtable}{llrrr}
\caption{Tablas de Frecuencia de la variables en estudio} \\ 
 \textbf{Variable} & \textbf{Levels} & $\mathbf{n}$ & $\mathbf{\%}$ & $\mathbf{\sum \%}$ \\ 
  \hline \hline
WorldFreedom & 1 & 55 & 26.7 & 26.7 \\ 
   & 3 & 62 & 30.1 & 56.8 \\ 
   & 5 & 89 & 43.2 & 100.0 \\ 
   \hline
 & all & 206 & 100.0 &  \\ 
   \hline
\hline
EconomicFreedom & 1 & 21 & 10.1 & 10.1 \\ 
   & 2 & 78 & 37.7 & 47.8 \\ 
   & 3 & 74 & 35.8 & 83.6 \\ 
   & 4 & 28 & 13.5 & 97.1 \\ 
   & 5 & 6 & 2.9 & 100.0 \\ 
   \hline
 & all & 207 & 100.0 &  \\ 
   \hline
\hline
PressFreedom & 1 & 22 & 10.7 & 10.7 \\ 
   & 2 & 53 & 25.7 & 36.4 \\ 
   & 3 & 66 & 32.0 & 68.5 \\ 
   & 4 & 48 & 23.3 & 91.8 \\ 
   & 5 & 17 & 8.2 & 100.0 \\ 
   \hline
 & all & 206 & 100.0 &  \\ 
   \hline
\hline
Democracy & 1 & 60 & 29.1 & 29.1 \\ 
   & 2 & 45 & 21.8 & 51.0 \\ 
   & 4 & 82 & 39.8 & 90.8 \\ 
   & 5 & 19 & 9.2 & 100.0 \\ 
   \hline
 & all & 206 & 100.0 &  \\ 
   \hline
\hline
\hline
\label{Tfrecuencias}
\end{longtable}
\endgroup

Como apreciamos en la Tabla \ref{Tfrecuencias}, los paÃ<U+00AD>ses en la mejor situación son los menos, salvo en el caso del \emph{Ã<U+00AD>ndice de libertas mundial}\footnote{Nótese que esto se puede deber a la {\bf menor} cantidad de categorÃ<U+00AD>as.}

\clearpage

Para resaltar lo anterior, tenemos la Figura \ref{barplots} en la página \pageref{barplots}. 


%%%%% figure
\begin{figure}[h]
\centering
\includegraphics{paperVersion_6-barplots}
\caption{Distribución de Indicadores}
\label{barplots}
\end{figure}

Además de la distribución de los variable, es importante saber el valor central. Como los valores son de naturaleza ordinal debemos pedir la {\bf mediana} y otras medidas de posición (como los \emph{cuartiles}, los que no pediremos pues son pocos valores). La mediana de cada variable la mostramos en la Tabla \ref{stats} en la página \pageref{stats}.

% Table created by stargazer v.5.2.2 by Marek Hlavac, Harvard University. E-mail: hlavac at fas.harvard.edu
% Date and time: mié, jul 25, 2018 - 08:36:15 p.m.
\begin{table}[!htbp] \centering 
  \caption{Medidas estadÃ<U+00AD>sticas} 
  \label{stats} 
\begin{tabular}{@{\extracolsep{5pt}}lcc} 
\\[-1.8ex]\hline 
\hline \\[-1.8ex] 
Statistic & \multicolumn{1}{c}{N} & \multicolumn{1}{c}{Median} \\ 
\hline \\[-1.8ex] 
WorldFreedom & 206 & 3.000 \\ 
EconomicFreedom & 207 & 3 \\ 
PressFreedom & 206 & 3.000 \\ 
Democracy & 206 & 2.000 \\ 
\hline \\[-1.8ex] 
\end{tabular} 
\end{table} 

\section{Exploración Bivariada}

En este trabajo estamos interesados en el impacto de los otros indices en el nivel de Democracia. Veamos las relaciones bivariadas que tiene esta variable con todas las demás:

% Table created by stargazer v.5.2.2 by Marek Hlavac, Harvard University. E-mail: hlavac at fas.harvard.edu
% Date and time: mié, jul 25, 2018 - 08:36:15 p.m.
\begin{table}[!htbp] \centering 
  \caption{Correlación de Democracia con las demás variables} 
  \label{corrDem} 
\begin{tabular}{@{\extracolsep{5pt}} ccc} 
\\[-1.8ex]\hline 
\hline \\[-1.8ex] 
WorldFreedom & EconomicFreedom & PressFreedom \\ 
\hline \\[-1.8ex] 
$0.896$ & $0.587$ & $0.771$ \\ 
\hline \\[-1.8ex] 
\end{tabular} 
\end{table} 

Veamos la correlación entre las variables independientes:


% Table created by stargazer v.5.2.2 by Marek Hlavac, Harvard University. E-mail: hlavac at fas.harvard.edu
% Date and time: mié, jul 25, 2018 - 08:36:15 p.m.
\begin{table}[!htbp] \centering 
  \caption{Correlación entre variables independientes} 
  \label{corrTableX} 
\begin{tabular}{@{\extracolsep{5pt}} cccc} 
\\[-1.8ex]\hline 
\hline \\[-1.8ex] 
 & WorldFreedom & EconomicFreedom & PressFreedom \\ 
\hline \\[-1.8ex] 
WorldFreedom & 1 &  &  \\ 
EconomicFreedom & 0.49 & 1 &  \\ 
PressFreedom & 0.83 & 0.53 & 1 \\ 
\hline \\[-1.8ex] 
\end{tabular} 
\end{table} 
Lo visto en la Tabla \ref{corrTableX} se refuerza claramente en la Figura \ref{corrPlotX}.

\begin{figure}[h]
\centering
\begin{adjustbox}{width=7cm,height=7cm,clip,trim=1.5cm 0.5cm 0cm 1.5cm}
\includegraphics{paperVersion_6-corrPlotX}
\end{adjustbox}
\caption{correlación entre predictores}
\label{corrPlotX}
\end{figure}


\clearpage

\section{Modelos de Regresión}

Finalmente, vemos los modelos propuestos. Primero sin la libertad mundial como independiente, y luego con está. Los resultados se muestran en la Tabla \ref{regresiones} de la página \pageref{regresiones}.



% Table created by stargazer v.5.2.2 by Marek Hlavac, Harvard University. E-mail: hlavac at fas.harvard.edu
% Date and time: mié, jul 25, 2018 - 08:36:15 p.m.
\begin{table}[!htbp] \centering 
  \caption{Modelos de Regresión} 
  \label{regresiones} 
\begin{tabular}{@{\extracolsep{5pt}}lcc} 
\\[-1.8ex]\hline 
\hline \\[-1.8ex] 
 & \multicolumn{2}{c}{\textit{Dependent variable:}} \\ 
\cline{2-3} 
\\[-1.8ex] & \multicolumn{2}{c}{Democracy} \\ 
\\[-1.8ex] & (1) & (2)\\ 
\hline \\[-1.8ex] 
 WorldFreedom &  & 0.704$^{***}$ \\ 
  &  & (0.046) \\ 
  & & \\ 
 EconomicFreedom & 0.377$^{***}$ & 0.291$^{***}$ \\ 
  & (0.077) & (0.053) \\ 
  & & \\ 
 PressFreedom & 0.833$^{***}$ & 0.012 \\ 
  & (0.065) & (0.070) \\ 
  & & \\ 
 Constant & $-$0.642$^{***}$ & $-$0.354$^{**}$ \\ 
  & (0.199) & (0.138) \\ 
  & & \\ 
\hline \\[-1.8ex] 
Observations & 206 & 206 \\ 
R$^{2}$ & 0.637 & 0.830 \\ 
Adjusted R$^{2}$ & 0.634 & 0.828 \\ 
Residual Std. Error & 0.880 (df = 203) & 0.603 (df = 202) \\ 
F Statistic & 178.197$^{***}$ (df = 2; 203) & 329.420$^{***}$ (df = 3; 202) \\ 
\hline 
\hline \\[-1.8ex] 
\textit{Note:}  & \multicolumn{2}{r}{$^{*}$p$<$0.1; $^{**}$p$<$0.05; $^{***}$p$<$0.01} \\ 
\end{tabular} 
\end{table} 
Como se vió en la Tabla \ref{regresiones}, cuando está presente el \emph{indice de libertad mundial}, el \emph{Ã<U+00AD>ndice de libertad de prensa} pierde significancia.

\clearpage

\section{Exploración Espacial}

Como acabamos de ver en la Tabla \ref{regresiones} en la página \pageref{regresiones}, si quisieras sintetizar la multidimensionalidad de nuestros indicadores, podrÃ<U+00AD>amos usar tres de las cuatro variables que tenemos (un par de las originales tiene demasiada correlación). 

AsÃ<U+00AD>, propongo que calculemos conglomerados de paÃ<U+00AD>ses usando toda la información de tres de los indicadores. Como nuestras variables son ordinales utilizaremos un proceso de conglomeración donde las distancia serán calculadas usando la medida {\bf gower}
propuesta por 
\cite{gower_general_1971}
, y para los enlazamientos usaremos la técnica de {\bf medoides} 
siguiendo a \cite{reynolds_clustering_2006}
. Los tres conglomerados se muestran en la Figura \ref{clustmap}.






\begin{figure}[h]
\centering
\begin{adjustbox}{width=11cm,height=8cm,clip,trim=1cm 2.5cm 0cm 2.5cm}
\includegraphics{paperVersion_6-plotMap1}
\end{adjustbox}
\caption{Paises conglomerados segun sus indicadores sociopolÃ<U+00AD>ticos}\label{clustmap}
\end{figure}

\bibliographystyle{apalike} %formato de estilo para la bibliografia- en APA es "apalike"
\renewcommand{\refname}{Bibliografia} %con renewcommand modificas el nombres predeterminado por ejemplo "Referencias" por "Bibliografía"
\bibliography{Repro}

\end{document}
